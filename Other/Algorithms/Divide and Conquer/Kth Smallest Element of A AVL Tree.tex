<span>QuickSelect</span> algorithm discussed in class is great when we
want to find the kth smallest element of an
unsorted array, but repeatedly calling this algorithm with different
values of k would be inefficient. We could solve this issue by sorting
the array, but this is only a good solution if the data does not change
(i.e. there are no insertions and deletions).

So suppose we have a collection of values that’s *dynamic*: we will be
inserting new values, and deleting old ones. In CPSC 221, you learned
that we could store this data in a balanced binary search tree (such as
an AVL tree) so every insertion and deletion operation takes in
O(log n) time, where n is the size of the tree. So assume that we
have a balanced binary search tree containing our data, and that in
addition to the usual fields each node also contains the size of its
*left subtree*.

Note that you can think of the `Quicksort` algorithm as implicitly
constructing a binary search tree: the pivot is the value stored at the
root of the tree, the array `Lesser` contains the elements of the left
subtree of the root, and `Greater` contains the elements of the right
subtree. So the root being aware of the size of its left subtree is like
`Quicksort` knowing the size of `Lesser`.

Describe an efficient algorithm that takes in such a tree T and a
positive integer k, and returns the kth
smallest element of T. What is the worst-case running time of your
algorithm?
```
Algorithm findKthSmallestElement(T,k){
    if(T.left_subtree_size == k-1){
        return T;
    }
    else{
        if(T.left_subtree_size < k-1){
            findKthSmallestElement(T.right, k-T.left_subtree_size);
        }
        else{
            if(T.left_subtree_size > k-1){
                findKthSmallestElement(T.left,k-T.left_subtree_size);
            }
        }
    }
}
```